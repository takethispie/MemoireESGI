\section*{Résumé}
Bien que le machine learning ait fait son apparition il y a plus de 50 ans,
ce n'est qu'aujourd'hui que l'interêt et les applications pour
l'intelligence artificelle ont explosés. Tandis que les cas d'utilisation 
évolue et que la question de l'automatisation des métiers se pose.
Nous etudions le paysage actuel de l'utilisation de l'intelligence 
artificielle dans des domaines professionnel ou non.
Alors que l'intelligence artificelle forte n'est que rêve digne des films 
de science fiction, l'intelligence artificelle faible est elle bien présente 
nous regardons pourquoi cette dernière ne sera pas capable d'avoir les même capacités 
cognitives que l'homme avec l'experience de pensée Chinese Room qui 
montre que la simulation de l'intelligence et la présence réelle de celle-Ci 
il y a une marge importante qui ne peut être comblée 
par l'intelligence artificelle faible, en étudiant les problèmes limitant les 
technologies actuelles nous proposons alors une visions possible du futur 
proche ou l'intelligence artificielle assiste l'homme pour lui 
permettre de se concentrer sur les tâches de créativité et d'innovation. 

\section*{Abstract}
Even though machine learning made it's appearance more than 50 years ago,
it is only today that interest in artificial intelligence and appplication
sky-rocketed. While use cases are constantly evolving we can ask ourselves 
the question of job automation.
We examine the current landscape of artificial intelligence usage in the professionnal
or personnal field.
Whereas strong artificial intelligence is only a sci-fi movie dream, weak 
artificial intelligence is real, we look why it won't be able to be as capable 
in cognitive tasks as us humans using the chinese room thinking experiment 
that shows simulating intelligence and actually showing proof of intelligence 
has a big gap between the two wich cannot be filled by weak artificial intelligence,
by examinating problems limiting current technologies we propose 
a vision of a possible futur where the artificial intelligence assist the individuals
to allow them to focus and more creative and innovative tasks.

\section*{keywords}
Machine learning; Artificial intelligence; Deep Learning; 
Job Market; Reinforced learning; explicability; interpretability;
intelligence artificelle; apprentissage profond; apprentissage renforcé;
marché du travail; apprentissage machine;

