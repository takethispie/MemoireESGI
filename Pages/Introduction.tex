\chapter*{Introduction}

En ce début de XXIème siècle, dans un monde où les Technologies de l'Information sont en constante évolution,
et où les entreprises cherchent en permanence de nouveaux moyens pour créer de la valeur
et optimiser le fonctionnement de leurs productions et de leurs services, l'attention se porte sur le rôle et la place de l'ordinateur au sein du fonctionnement de notre Société.\newline

Ayant repéré et analysé la plupart des problèmes qui peuvent être rencontrés lors de la réalisation de tâches concrètes, l'Humain d'aujourd'hui
porte son attention en particulier sur la quantité significative d'erreurs dont il peut faire preuve.
L'Erreur étant une notion inhérente à l'Humain, ce dernier cherche en priorité à l'éliminer le plus possbile de ses réalisations.\newline

Additionnellement, chaque individu se soucie en parallèle de sa santé, que ce soit concernant son propre corps ou de son esprit,
et actuellement certaines carrières posent des contraintes vis à vis de cet aspect.
Ainsi, nous, individus, nous nous rendons compte au fur et à mesure du temps des contraintes physiques de notre corps, ce qui nous incite à trouver des
idées afin de repousser encore et encore ces limites.\newline

C'est ainsi pour ces différentes raisons que les regards se portent désormais sur les progès de la Science, notamment dans le domaines des technologies de l'Information,
et plus particulièrement dans le domaine de l'"Intelligence Artificielle".\newline

Déjà présentes au jour d'aujourd'hui dans de nombreux foyers, et cela sous de diverses formes allant de nos smartphones à nos haut-parleurs intelligents,
en passant par les services en ligne de communication par chat ou de service après-vente, l'IA est en passe de s'ancrer encore plus profondément dans nos vies
avec l'implémentation des différents outils qu'elle propose de nos jours. L'étendue des possibilités d'utilisations de cette dernière est vaste.\newline

Mais l'utilisation plus poussée de l'IA, d'un tel changement dans nos vies, n'est pas sans conséquences. En effet, si l'IA est devenue plus efficace que l'humain sur certaines tâches,
alors pourquoi les entreprises ne poseraient-elles pas le question suivante : Est-ce que je peux remplacer totalement un employé spécialisé, voire une partie de mon effectif de spécialistes par une IA?
À l'aube des prochaines innovations informatiques allant dans le sens de ces innovations, de ces évolutions du quotidien des travailleurs à prévoir, il parait naturel de se poser la question suivante :\newline

Comment l'IA aura-t-elle un réel impact sur le marché du travail dans les années à venir ?

Afin d'apporter des éléments de réponse à cette question, nous nous intéresserons d'abord à la façon dont l'IA remplace 
l'Humain dans des tâches précises et répétitives.
Par la suite, nous verrons pourquoi l'intelligence ne peux pas remplacer l'homme pour toutes les tâches.
Puis nous concluerons par l'explication des scénarios que nous considerons les plus probables dans un avenir proche.
