\part{L'humain se concentre sur les tâches qui nécéssite d'avoir des traits humains}
\chapter{l'Intelligence Artificelle ne peux pas remplacer l'humain pour toutes les tâches}
\section{Intelligence Artificielle Forte}
L'intelligence artificelle forte est l'intelligence
telle qu'elle existe chez l'homme, une somme de procédés cognitifs avancés mais
même aujourd'hui le fonctionnement du cerveau et de l'intelligence reste mystérieuse
et donc la faisabilité d'une IA forte est sans cesse remise en question.

\subsection{Prérequis}
puisque la définition de l'Intelligence elle même reste flou, il est difficile 
de donner une liste exacte et correcte des critères pour qu'une IA forte 
puisse exprimer une intelligence semblable à celle de l'homme mais 
il une liste de critères semble être indéniablement nécéssaires pour remplir 
les critères et la majorité des chercheurs en intelligence artificelle semble 
s'être mis d'accord sur la liste de critères suivantes: 

%TODO approfondir chaque item
\begin{itemize}
    \item Capacité de raisonnement et de jugement
    \item Capacité à conceptualiser ses connaissances 
    \item Capacité de communication dans un langage naturel (langue humaine)
    \item Capacité de planification
\end{itemize}


\subsection{Freins majeur de la création d'Intelligence Artificelle Forte}
%la meta cognition est un des freins majeur (la cognition sur la cognition)
\newpage

%le cheminement de pensee de cette partie est d'expliquer cette experience 
%pour montrer qu'une IA ne peu pas actuellement etre intelligente comme un humain
%et utiliser la conclusion de cette section sur lequel ce basera pour 
%eliminer la possibilité de remplacement de l'humain dans le futur proche 
%dans les métiers qui necessitent de l'intelligence humaine
%
%un contre argument en particulier de cette experience peut etre partiellement 
%invalidé: celui de Douglas Hofstadter qui dit que les règles syntaxiques à 
%elle seule ne suffisent pas à elle seule pour reproduire la 
%compréhension du langage il faut comprendre le monde 
%
%mon contre argument est justement d'associer le manuel utilisé dans la piece fermé
%associé à un algorithme de deep learning entrainé sur des millions de data de 
%conversation, pour avoir la partie syntaxique (homme+manuel) 
%et la partie sémantique (deep learning)
%
%en soit le deep learning en lui meme invalide le contre argument
%exemple: personnal assistants etc mais seulement dans des cas très précis
%puisque les assistant ne passerait pas le test de turing 
\section{L'experience de pensée "Chinese Room"}
En 1980 John Searle, philosophe américain, publie son article "Minds, Brains, and Programs" dans la revue 
scientifique "Behavioral and Brain Sciences" qui donna lieu à de grands débat dans le domaine philosophique 
mais surtout dans le domaine de l'intelligence artificelle, 
la source de ces débats est une expérience de pensée qui se nomme "Chinese Room". \newline

\begin{figure}[!h]
    \centering
    \includegraphics[width=0.8\textwidth]{Images/chineseroom}
    \caption{Chinese room experiment - wikicommons}
	\label{fig:chineseroom}
\end{figure}

Cette expérience est définie comme suit: \newline
il y a, enfermé dans une piece sans aucun moyen de contact vers l'exterieur, une personne anglophone qui ne comprend 
pas le chinois et dans cette piece des boites remplies de symboles chinois ainsi qu'un manuel d'instructions.
Cette personne reçois des symboles chinois envoyé par une personne parlant chinois qui sont en réalité des questions,
dans le manuel d'instruction est indiqué quoi renvoyer en fonction de ce que la personne anglophone reçois,
la personne renvoie des symboles qui sont des réponses à la question reçue, la personne parlant le chinois
pense ainsi parler à une personne qui connaît la langue alors que ce n'est pas le cas. \newline 

l'argument de cette expérience est que meme si la machine répond aux questions qui semble laisser penser 
la présence de capacité à penser (une IA forte) elle ne fait en réalite que manipuler des symboles 
sans les interpreter (IA faible). \newline

l'objectif de cette experience est d'invalider la capacité du test du turing à établir si une 
intelligence artificelle a la capacité de penser, le test de turing est une expérience à l'aveugle 
ou une personne converse avec un interlocuteur qui est est soit une vrai personne ou un humain 
si le sujet avec qui l'IA converse n'est pas capable de detecter qu'il ne parle pas avec un humain mais
une machine cette dernière réussie le test. 

%explication en profondeur de la chinese room

\subsection{L'argument de la reproduction de l'intelligence contre la conscience intentionnelle}

%montrer que l'IA ne peut pas avoir de conscience et donc la meme intelligence que l'humain, 
%ce qui veut dire que l'humain ne pourra être remplacé que dans les taches non repetitives 

%ce qui suit sont des notes de reflexion sans aucun sens 
% lhomme apprend à la machien = on injecte en quelquesorte la projection de notre conscience 
%on utilise la langue pour communiquer ce que l'on veut or notre langue = sementique => "syntax does not 
%suffice for semantic"  
%
%compiler = grammar rule + regle semantic écrite à la main => une infinité de regles pour l'esprit en plus 
%du bon sens commun 

%le livre est écrit par un humain, et l'anglais n'est qu'un intermediaire => l'ecrivain du livre aka 
%programmeur doit savoir parler chinois = est ce l'on parle au programmeur au final ? 

%And the reason is that 
%the entire system, me, symbols, rule books, room, etc., contain only Chinese symbolic devices but no meanings
%le langage par lui meme est symbolique et n'est qu'un support physique du meaning qu´on leur donne




\subsection{Application à la problématique de l'automatisation des métiers}

\part{Vers une synergie homme-machine}