\part{L'humain se concentre sur les tâches qui nécéssite d'avoir des traits humains}
\chapter{l'Intelligence Artificelle ne peux pas remplacer l'humain pour toutes les tâches}
\section{Intelligence Artificielle Forte}
L'intelligence artificelle forte est l'intelligence
telle qu'elle existe chez l'homme, 

%placeholder, peut etre temporaire
\subsection{Prérequis}
\subsection{Freins majeur de la création d'Intelligence Artificelle Forte}

%je ne sais pas encore si je garde c'est partie elle est extremement complexe
%le cheminement de pensee de cette partie est d'expliquer cette experience 
%pour montrer qu'une IA ne peu pas actuellement etre intelligente comme un humain
%et utiliser la conclusion de cette section sur lequel ce basera pour 
%eliminer la possibilité de remplacement de l'humain dans le futur proche 
%dans les métiers qui necessitent de l'intelligence humaine
%
%un contre argument en particulier de cette experience peut etre partiellement 
%invalidé: celui de Douglas Hofstadter qui dit que les règles syntaxiques à 
%elle seule ne suffisent pas à elle seule pour reproduire la 
%compréhension du langage il faut comprendre le monde 
%
%mon contre argument est justement d'associer le manuel utilisé dans la piece fermé
%associé à un algorithme de deep learning entrainé sur des millions de data de 
%conversation, pour avoir la partie syntaxique (homme+manuel) 
%et la partie sémantique (deep learning)
%
%en soit le deep learning en lui meme invalide le contre argument
%exemple: personnal assistants etc mais seulement dans des cas très précis
%puisque les assistant ne passerait pas le test de turing 
\section{L'experience de pensée "Chinese Room"}
\subsection{L'argument de la reproduction de l'intelligence contre la conscience intentionnelle}
\subsection{Application à la problématique de l'automatisation des métiers}

\chapter{Vers une synergie homme-machine}