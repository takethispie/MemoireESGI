\part{L'humain se concentre sur les tâches qui nécéssite d'avoir des traits humains}
\chapter{l'Intelligence Artificelle ne peux pas remplacer l'humain pour toutes les tâches}
\section{Intelligence Artificielle Forte}
L'intelligence artificelle forte est l'intelligence
telle qu'elle existe chez l'homme, une somme de procédés cognitifs avancés mais
même aujourd'hui le fonctionnement du cerveau et de l'intelligence reste mystérieuse
et donc la faisabilité d'une IA forte est sans cesse remise en question.

\subsection{Prérequis}
puisque la définition de l'Intelligence elle même reste flou, il est difficile 
de donner une liste exacte et correcte des critères pour qu'une IA forte 
puisse exprimer une intelligence semblable à celle de l'homme mais 
il une liste de critères semble être indéniablement nécéssaires pour remplir 
les critères et la majorité des chercheurs en intelligence artificelle semble 
s'être mis d'accord sur la liste de critères suivantes: 

%TODO approfondir chaque item
\begin{itemize}
    \item Capacité de raisonnement et de jugement
    \item Capacité à conceptualiser ses connaissances 
    \item Capacité de communication dans un langage naturel (langue humaine)
    \item Capacité de planification
\end{itemize}


\subsection{Freins majeur de la création d'Intelligence Artificelle Forte}
%la meta cognition est un des freins majeur (la cognition sur la cognition)
\newpage

%je ne sais pas encore si je garde c'est partie elle est extremement complexe
%le cheminement de pensee de cette partie est d'expliquer cette experience 
%pour montrer qu'une IA ne peu pas actuellement etre intelligente comme un humain
%et utiliser la conclusion de cette section sur lequel ce basera pour 
%eliminer la possibilité de remplacement de l'humain dans le futur proche 
%dans les métiers qui necessitent de l'intelligence humaine
%
%un contre argument en particulier de cette experience peut etre partiellement 
%invalidé: celui de Douglas Hofstadter qui dit que les règles syntaxiques à 
%elle seule ne suffisent pas à elle seule pour reproduire la 
%compréhension du langage il faut comprendre le monde 
%
%mon contre argument est justement d'associer le manuel utilisé dans la piece fermé
%associé à un algorithme de deep learning entrainé sur des millions de data de 
%conversation, pour avoir la partie syntaxique (homme+manuel) 
%et la partie sémantique (deep learning)
%
%en soit le deep learning en lui meme invalide le contre argument
%exemple: personnal assistants etc mais seulement dans des cas très précis
%puisque les assistant ne passerait pas le test de turing 
\section{L'experience de pensée "Chinese Room"}
\subsection{L'argument de la reproduction de l'intelligence contre la conscience intentionnelle}
\subsection{Application à la problématique de l'automatisation des métiers}

%passr ça en 3 ieme partie
\part{Vers une synergie homme-machine}