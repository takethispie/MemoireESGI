\part{Vers une synergie homme-machine}
    Alors que l'intelligence artificelle a un penchant "buzzword" utilisé par des entreprises pour 
    mieux vendre leur produit en faisant l'association IA égal à plus de performances, et que les 
    site et média d'informations montre du doigt l'intelligence artificelle comme le futur destructeur 
    d'emploi, il est important de souligner que les êtres humains sont passé par de telles phases 
    auparavant: \newline 

    \begin{itemize}
        \item La révolution agricole:En Europe, pendant le 18ième siècle moins de terres en jachère,
        << En 1840, 7 millions d'hectares c'est-à-dire 25\% 
        des terres européennes sont en jachère ; en 1900, il n'y a plus que 3
         millions d'hectares en jachère, c'est-à-dire 10 \% des terres. >>
         \footnote{source: \url{https://www.philisto.fr/cours-49-transformations-economiques-de-l-europe-xixe-siecle.html}}
         l'apparition de la moissonneuse et la batteuse à vapeur changent les méthodes agricoles 
         vers une agriculture orienté capitalisme. 
         Cette transformation a été globalement avantageuse pour tout le peuple, l'utilisation 
         des machines étant rare à cause de leur prix, l'utilisation d'animaux de traits ne 
         disparaitra que très progressivement ne laissant pas sans capacité de travail les personnes
         qui dépendaient de leur utilisation dans l'agriculture (la mecanisation ne s'effectuera 
         massivement qu'a partir de la fin de la seconde guerre mondiale). \newline 

         \item La révolution industrielle: c'est la transformation qui aura eu le plus grand impact,
         elle est decomposé en deux temp qui peuvent globalement être associé avec 
         l'utilisation de la machine à vapeur puis l'apparition de l'electricité, du gaz ainsi que 
         moteur à explosion, l'exode rural qui s'ensuit (fuir la campagne et son agriculture pour 
         rejoindre la ville et ses usines) et dû à une jeunesse formé à l'agriculture 
         qui ne voit pas de futur dans l'agriculture et une opportunité dans l'industrie,
         cette période montre que l'homme et les nations ont les capacités pour s'adapter 
         aux paysages économiques en perpétuels changement, mais les challenges à surmonter 
         avec l'intelligence artificelle sont différents puisqu'elle va et a déjà bousculée tout 
         les domaines y compris l'agriculture et l'industrie, malgré cela pour que cette 
         révolution soit aussi disruptives il y a de nombreux challenges à dépasser.
    \end{itemize}


    \chapter{Surmonter les problèmes inhérent au machine learning }
        Des problèmes empecheront l'intelligence artificelle notamment à cause de l'utilisation 
        du machine learning, technologie qui va rester d'actualité pendant de nombreuses années car nous
        avons qu'effleuré la surface et les possibilités de celle-ci.
        Tout d'abord le principe d'explicabilité, la difficulté de reproductibilité et enfin l'applicabilité 
        métiers sont les trois principaux points bloquants pour une évolution positive de 
        l'intelligence artificielle. \newline

        \section{Explicabilité}
            Une des raison de la resistance au changement est l'incapacité à comprendre 
            le fonctionnement interne d'une intelligence artificelle utilisant le machine learning, 
            ce qui rend une entreprise dépendant de cette dernière avec un maîtrise relative
            entre les mains des data scientists. \newline   


        \section{Reproductibilité}


        \section{Applicatibilité Métier}
            Cette problématique concerne l'identification des besoins auquels l'intelligence artificelle 
            peut répondre, comme indiqué dans la partie précèdente une des principales difficulté réside 
            dans l'application concrète de celle-ci, à quel(s) besoin(s) peut répondre une 
            intelligence artificelle qu'un algorithme "traditionnel" ne peut pas ? 



        %utilisation humain va shifter dans le luxe et petites serie dans l'industrie 
        %un peu comme les voitures  monté à la main 
        %un effet d'elite 
        %l'IA va devenir le "bas de gamme" de la création de bien 
        %tandis que les industries et à echelle humaine vont rentrer dans le luxe ? 
        %le luxe ne "subis pas la crise"

        %services utilisant l'IA pour desservir les endroit qui on peu ou pas accès 
        %à lesdit services (ex: campagne )
    \chapter{Automatiser les taches sans grandes valeurs ajoutée}
        
        %bonne idée ou pas ?
        \section{La valeur intrinsèque de métiers qui ne semblent pas en avoir }
        \section{Re-spécialisiation au sein des métiers automatisés}
        
        %explicabilité des IA: les IA qui détail leur "reflexion" pour arriver à leur résultat 
        % => moins de resistance au changement 


    \chapter{Utiliser l'IA comme un assistant de productivité}
