\chapter*{Conclusion}
L'Intelligence Artificielle telle que nous la connaissons, va remplacer les emplois répétitifs requérant peu de réflexion et
à peu de valeur ajoutée. Cela représenterait 9\% des emplois en France. \newline

Elle impactera également les emplois à plus forte valeur ajoutée en proposant des aides décisionnelles,
cela reviendrait au concept d'Intelligence Augmentée. En réponse à la demande croissante, de nombreux postes seront créés dans le domaine
de l'Informatique et de la robotique, dans le but de gérer le matériel mais aussi les logiciels ainsi que la sécurité de l'ensemble. \newline

Les employés ayant perdu leurs postes devront effectuer une reconversion dans des domaines d'activité liés
aux relations interhumaines, créatifs, artisanaux ou devront se former pour travailler à un niveau plus décisionnel. \newline

L'Education Nationale et l'Éducation Supérieure devront prendre en compte
ces changements sociétaux et s'adapter pour proposer une formation plus adéquate et une orientation vers des métiers moins impactés par
l'Intelligence Artificielle. L'État devra proposer des formations pour la reconversion professionnelle afin de rediriger les personnes actives
vers des branches proposant des emplois plus nombreux et moins menacés.
Cette période de transition devra être progressive, car plus elle sera rapide et brutale, plus elle sera
mal vécue et rejetée par les personnes les plus concernées par les changements induits. \newline

Cela étant, toute cette transition pourrait être elle-même bouleversée par l'arrivée d'une Intelligence
Artificielle considérée comme "Forte" en comparaison aux IAs dites "Faibles" telles que nous les connaissons aujourd'hui. L'avancée
serait de taille, puisque ce type d'intelligence serait capable d'appréhender ses propres raisonnements et de développer une certaine
conscience, s'approchant ainsi encore plus d'une véritable intelligence.