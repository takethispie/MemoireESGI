\chapter*{Conclusion}
L'intelligence artificielle dites "forte" n'étant pas réalisable avec 
les techniques actuelles même à but purement académique 
et ne pouvant être créee à partir d'une intelligence artificielle dites "faible"
c'est cette dernière qui va rester au centre des avancées technologiques.
L'IA va redéfinir nombre de métiers replacant les tâches répétitives des métiers 
pour laisser place à de nouvelles voir même prendre la place d'humain pour certains 
métiers tel que le transport et la livraison, les domaines nécéssitant 
des traits humain tel que la créativité ou l'empathie seront plus difficilement 
automatisable, de manière générale l'intelligence artificielle va avoir 
un effet disruptif sur l'economie mondiale dans les années à venir 
en s'immiscant dans tout les domaines, ils est du ressort des instutions
d'encadrer correctement pour tirer le plus de positif de cette révolution
qui pourrait, dans les années futur surpasser toutes les révolutions économiques
et sociales, le risque majeur étant une destruction des emplois sans re-spécialisation
et un écart entre les personnes faiblement diplômées et les personnes 
hautement diplômées encore plus grand.
